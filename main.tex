%---------------------------------------------------------------------------
% scrlttr2.tex v0.3. (c) by Juergen Fenn <juergen.fenn@gmx.de>
% Template for a letter to be typeset with scrlttr2.cls from KOMA-Script.
% Latest version of the LaTeX Project Public License is applicable. 
% File may not be modified and redistributed under the same name 
% without the author's prior consent.
%---------------------------------------------------------------------------
\documentclass%%
%---------------------------------------------------------------------------
  [fontsize=12pt,%%          Schriftgroesse
%---------------------------------------------------------------------------
% Satzspiegel
   paper=a4,%%               Papierformat
   enlargefirstpage=on,%%    Erste Seite anders
   pagenumber=headright,%%   Seitenzahl oben mittig
%---------------------------------------------------------------------------
% Layout
   headsepline=on,%%         Linie unter der Seitenzahl
   parskip=half,%%           Abstand zwischen Absaetzen
   %sections,
%---------------------------------------------------------------------------
% Briefkopf und Anschrift
   fromalign=right,%%        Plazierung des Briefkopfs
   fromphone=on,%%           Telefonnummer im Absender
   fromrule=off,%%           Linie im Absender (aftername, afteraddress)
   fromfax=off,%%            Faxnummer
   fromemail=off,%%          Emailadresse
   fromurl=off,%%            Homepage
   fromlogo=off,%%           Firmenlogo
   addrfield=on,%%           Adressfeld fuer Fensterkuverts
   backaddress=on,%%          ...und Absender im Fenster
   subject=beforeopening,%%  Plazierung der Betreffzeile
   locfield=narrow,%%        zusaetzliches Feld fuer Absender
   foldmarks=on,%%           Faltmarken setzen
   numericaldate=off,%%      Datum numerisch ausgeben
   refline=narrow,%%         Geschaeftszeile im Satzspiegel
%---------------------------------------------------------------------------
% Formatierung
   draft=false%%                Entwurfsmodus
]{scrlttr2}
%---------------------------------------------------------------------------
\usepackage{ngerman}
\usepackage[T1]{fontenc}
\usepackage[latin1]{inputenc}
\RequirePackage{graphicx}
\usepackage{url} 
\usepackage{eurosym} 
\usepackage{advdate} 
% Use URW Garamond No. 8 as a default font. (getnonfreefonts)
\renewcommand{\rmdefault}{ugm}
% Optima as a sans serif font.
\renewcommand*\sfdefault{uop}
\usepackage[protrusion=true,expansion=true]{microtype}
\clubpenalty=3000 % adjust for widows and orphans 10000 is max
\widowpenalty=3000 % adjust for widows and orphans 10000 is max
%---------------------------------------------------------------------------
% Fonts
\setkomafont{fromname}{\sffamily \LARGE}
\setkomafont{fromaddress}{\sffamily}%% statt \small
\setkomafont{pagenumber}{\sffamily}
\setkomafont{subject}{\mdseries}
\setkomafont{backaddress}{\sffamily}
%\usepackage{mathptmx}%% Schrift Times
%\usepackage{mathpazo}%% Schrift Palatino
%\setkomafont{fromname}{\LARGE}
%---------------------------------------------------------------------------
\begin{document}
%---------------------------------------------------------------------------
% Briefstil und Position des Briefkopfs
\LoadLetterOption{DIN} %% oder: DINmtext, SN, SNleft, KOMAold.
\makeatletter
\@setplength{firstheadvpos}{5mm}
\@setplength{firstheadwidth}{\paperwidth}
\ifdim \useplength{toaddrhpos}>\z@
  \@addtoplength[-2]{firstheadwidth}{\useplength{toaddrhpos}}
\else
  \@addtoplength[2]{firstheadwidth}{\useplength{toaddrhpos}}
\fi
\@setplength{foldmarkhpos}{6.5mm}
\makeatother
%---------------------------------------------------------------------------
% Absender
\setkomavar{fromname}{Friedrich Nord}
\setkomavar{fromaddress}{Lolwut 2\\27182 Bielefeld}
\setkomavar{fromphone}{+49 (0) 678 2342}
\setkomavar{fromemail}{f.nord@cheezburge.rz}
\setkomavar{fromurl}{http://cheezburge.rz}
\setkomavar{signature}{F. Nord}
%\setkomavar{fromlogo}{\large{\textbf{\sffamily{Friedrich Nord}}}}
\renewcommand{\phonename}{Telefon}
\setkomavar{backaddressseparator}{, }
%\setkomavar{frombank}{}
%\setkomavar{location}{\\[8ex]\raggedleft{\footnotesize{\usekomavar{fromaddress}\\
%      Telefon:\ usekomavar{fromphone}}}}%% Neben dem Adressfenster
%---------------------------------------------------------------------------
% Briefkopf
\firsthead{
%\begin{tabular}[h]{lr}
%  \hspace{15.5cm}&
%  \begin{minipage}[t]{.95\textwidth} 
    \raggedleft
    {
    {\large{\textbf{\sffamily{\usekomavar{fromname}}}}}\\ 
    \mdseries\vspace{2mm}
    \usekomavar{fromaddress}\\
    Tel. \usekomavar{fromphone}\\ 
    \usekomavar{fromemail}
    }
%  \end{minipage}
%\end{tabular}
}
%---------------------------------------------------------------------------
\firstfoot{\tiny{Proudly made using \LaTeX, VIM, Make \& Coffee.}}
%---------------------------------------------------------------------------
% Geschaeftszeilenfelder
\setkomavar{place}{Bielefeld}
\setkomavar{placeseparator}{, den }
\setkomavar{date}{\today}
%\setkomavar{yourmail}{24.06.2010}%% 'Ihr Schreiben...'
%\setkomavar{yourref} {S1/Cm-Vm}%%    'Ihr Zeichen...'
%\setkomavar{myref}{}%%      Unser Zeichen
%\setkomavar{invoice}{123}%% Rechnungsnummer
%\setkomavar{phoneseparator}{}
%---------------------------------------------------------------------------
% Versendungsart
%\setkomavar{specialmail}{Einschreiben mit R�ckschein}
%---------------------------------------------------------------------------
% Anlage neu definieren
\renewcommand{\enclname}{Anlage}
\setkomavar{enclseparator}{: }
%---------------------------------------------------------------------------
% Seitenstil
\pagestyle{plain}%% keine Header in der Kopfzeile
%---------------------------------------------------------------------------
\begin{letter}{
Teh Evil Overlord\\
blah 2\par
12345 blubb
}
%---------------------------------------------------------------------------
% Weitere Optionen
\KOMAoptions{%%
}
%---------------------------------------------------------------------------
\setkomavar{subject}{Antrag auf Auskunftserteilung gem�� �34 BDSG sowie Widerspruch
nach �24(4) BDSG}
%---------------------------------------------------------------------------
\opening{Sehr geehrte Damen und Herren,}

ich habe am 19.05.2011 bei Ihnen �bernachtet. Da sie dabei
personenbezogene Daten �ber mich verarbeitet haben,
bitte ich Sie um Auskunft �ber alle
zu meiner Person bei Ihnen gespeicherten Daten gem�� �34 BDSG:

\begin{enumerate}
  \item Welche personenbezogenen Daten �ber mich werden von Ihrem 
    Unternehmen gespeichert? Hierbei bitte ich Sie um Auskunft �ber 
    die gespeicherten Daten selbst und nicht 
    nur �ber die Art von Daten, die Sie gespeichert haben.
  \item Falls ich Ihnen diese Daten nicht selbst mitgeteilt habe: Woher 
    und zu welchem Zeitpunkt hat Ihr Unternehmen diese Daten bezogen?
  \item An welche Empf�nger --- auch innerhalb Ihrer Firmengruppe ---
    wurden oder werden diese Daten durch Ihr Unternehmen weitergegeben?
  \item Zu welchen Zwecken speichert Ihr Unternehmen diese Daten?
\end{enumerate}

Ich bitte Sie, die Auskunft bis zum
\begin{center}
  \DayAfter[15]
\end{center}
zu erteilen. Bitte best�tigen Sie mir kurz den Eingang dieser Nachricht und
senden Sie mir die Auskunft \emph{postalisch} zu. Sollten Sie --- wider meiner
Erwartung --- dieses Schreiben ignorieren, so werde ich mich an den
zust�ndigen Landesdatenschutzbeauftragten wenden. Ausserdem behalte ich
mir in diesem Fall weitere rechtliche Schritte vor.

Weiterhin nutze ich diese Gelegenheit, der Weitergabe und Nutzung meiner
Daten f�r Zwecke der Werbung oder der Markt- oder Meinungsforschung zu
widersprechen (�28 Absatz 4 BDSG). Sie sind daher verpflichtet, die
Daten unverz�glich f�r diese Zwecke zu sperren.  Ebenso widerspreche ich
der Weitergabe meiner Daten an andere Unternehmen, auch \emph{innerhalb} Ihrer
Firmengruppe.

\closing{Mit freundlichen Gr��en,\\
%\vspace{1cm}\includegraphics[width=5cm]{md-signature.jpg}
}
%---------------------------------------------------------------------------
\ps{PS:
Ich kann verstehen, dass mein Auskunftsersuchen Ihnen Arbeit
verursacht. Eine Alternative w�re es, den \emph{Datenbrief}
umzusetzen und einmal im Jahr bei einem Kundenkontakt (z.B. der
Werbesendung, die Sie mir geschickt haben) �ber die bei Ihnen
gespeicherten Kundendaten zu informieren. Dies w�rde erheblich dazu beitragen,
Ihre Datenverarbeitung transparent und im Sinne des Kunden zu gestalten.
Einen Einstieg in dieses Thema finden Sie hier:

\begin{center}
  \url{http://de.wikipedia.org/wiki/Datenbrief}
\end{center}

Mit diesem Vorgehen k�nnten Sie einerseits Ihre Arbeitslast durch
Auskunftsersuchen wie meines reduzieren und andererseits einen
Mehrwert f�r Ihre Kunden schaffen. Gleichzeitig k�nnten Sie auch
Ihre Stammdaten auf Ihre Aktualit�t �berpr�fen.  }
%\encl{} \cc{}
%---------------------------------------------------------------------------
\end{letter}
%---------------------------------------------------------------------------
\end{document}
%---------------------------------------------------------------------------

% vim: set spell spelllang=DE setf tex :EOF
